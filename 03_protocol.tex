\section{Experimental protocol}
\label{sect:protocol}
\subsection{Datasets}

In the current literature, there is no audio data annotated to perform SAD, OSD, MD, and ND simultaneously.
Therefore, models are trained on several datasets to perform multilabel audio segmentation so that all the classes are represented.
The teacher is trained on 4 datasets listed in the table \ref{tab:datasets}.
Due to the large amount of data, the teacher pre-training requires a lot of resources and a lengthy training process to converge.
We propose to train the proxy model on a subset of the training set.
Only Albayzin and DiHard data are used for the knowledge distillation step.

\begin{table}[hb]
    \centering
    \begin{tabular}{lcccccc}
         \toprule
         & \multicolumn{2}{c}{Model} & \multicolumn{4}{c}{Annotation} \\
         \cmidrule{2-3}
         \cmidrule{4-7}
         Dataset & T & P & SAD & MD & ND & OSD \\
         \midrule
         Albayzin \cite{albayzin10,albayzin12} & \checkmark & \checkmark & \checkmark & \checkmark & \checkmark &  \\
         OpenBMAT \cite{openBMAT} & \checkmark & & & \checkmark & & \\
         ALLIES \cite{larcher:hal-03262914_short} & \checkmark & & \checkmark & & & \checkmark \\
         DiHard III \cite{ryant2021dihard} & \checkmark & \checkmark & \checkmark & & & \checkmark \\
         \bottomrule
    \end{tabular}
    \caption{Datasets used to train both teacher (T) and proxy (P) models with the available labels in each of them.}
    \label{tab:datasets}
\end{table}

\subsection{Model architectures}

The teacher model is similar to \cite{lebourdais22_interspeech} and is composed of two main parts: feature extraction and sequence modeling.
The former is performed using Large Wavlm pre-trained features \cite{chen2022wavlm}.
The obtained output is a sequence of vectors of $D=1024$ dimensions.
The latter part of the model inputs the sequence of Wavlm features and applies transformations to predict the segmentation.
A bottleneck layer of 64 channels first transforms these features.
3 TCN blocks process the output of the bottleneck layer \cite{bai_empirical_2018} composed of 5 1-d convolutional layers with exponentially increasing dilatation. 
A 1-d convolution layer processes the output of the TCN blocks to project the hidden representations to the logits space.
A sigmoid activation function is then applied to get normalized scores for each class.

As described in section \ref{}, two types of proxy models are investigated.
In the first approach, the proxy model inputs a spectrogram extracted on 64ms sliding windows with 20ms step.
It is then processed by 4 block TCN models where each block comprises 4 layers.
The bottleneck and hidden layer are composed of 128 and 256 channels respectively.
The second approach uses Wavlm features from the teacher as input.
In this case, the TCN model is smaller, with 2 blocks of 3 1-d convolution layers. 
The bottleneck and hidden layer are composed of 64 and 128 channels respectively.

\subsection{NMF pre-training}

As previously introduced, the $\mathbf{W}$ dictionary is pre-trained to map the embedding space to the spectrogram space.
Two approaches were investigated.
First, we allowed the $\mathbf{W}$ dictionary to be trained with the proxy model.
In this case, the dictionary is implemented as a linear layer with no bias.
Second, the dictionary is pre-trained with the standard NMF approach and fixed during the proxy model training.
The pre-training is performed on a subset of Aragon Radio.
We select segments between 1s and 4s representing each type of class.
In practice, the subset is built with 1200 segments with 16\% of speech and 42\% of music and noise.
For each approach, the NMF rank is fixed at $K=256$.


\subsection{Evaluation procedure}

To evaluate the segmentation performance, both teacher and proxy models are evaluated following the same protocol.
The models are evaluated on Aragon Radio and DiHard III test sets.
The test partitions follow the split proposed for the original challenges \cite{albayzin12,ryant2021dihard}.
